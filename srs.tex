\documentclass{scrreprt}
\usepackage{listings}
\usepackage{underscore}
\usepackage[T2A]{fontenc}
\usepackage[utf8]{inputenc}
\usepackage[russian]{babel}
\date{}
%\title
\usepackage{hyperref}
\begin{document}

\tableofcontents

\chapter{Introduction}

\section{Purpose}
Целью данного документа является предоставление подробного описания требований к программному обеспечению «РБК Недвижимость». Он проиллюстрирует цель, объяснит системные ограничения, интерфейс и взаимодействие с другими
внешними приложениями. Этот документ в первую очередь предназначен для предложения заказчику и в качестве справочного материала для команды разработчиков.

\section{Project Scope}
Документ относится к разрабатываемому веб-сайту https://realty.rbc.ru/ - аналитическому и новостному порталу о недвижимости, объединяющий актуальные новости, рыночные данные. Описанная в этом документе система позволит пользователям использовать сайт для просмотра и публикации статей и новостей о недвижимости.

\section{Definitions, Acronyms and Abbreviations}
$<$Describe any standards or typographical conventions that were followed when 
writing this SRS, such as fonts or highlighting that have special significance.  
For example, state whether priorities  for higher-level requirements are assumed 
to be inherited by detailed requirements, or whether every requirement statement 
is to have its own priority.$>$

\section{Intended Audience and Reading Suggestions}
Данный документ предназначен для:
\begin{itemize}
	\item Руководителя проекта
	\item Разработчиков
	\item Тестировщиков
	\item Сотрудников отдела маркетинга
	\item Дизайнеров и UX-аналитиков
\end{itemize}

\section{References}
$<$List any other documents or Web addresses to which this SRS refers. These may 
include user interface style guides, contracts, standards, system requirements 
specifications, use case documents, or a vision and scope document. Provide 
enough information so that the reader could access a copy of each reference, 
including title, author, version number, date, and source or location.$>$

\section{Overview}
Следующий раздел, Overall Description, представит общее описание разрабатываемого
продукта. В частности, опишет предполагаемый функционал продукта, различные
ограничения и зависимости на высоком уровне.
Третий раздел, Specific Requirements, содержит конкретное описание различных функций,
требований к удобству, безопасности и надежности, необходимых при разработке системы.
Также в нем указаны технологии, системы разработки и прочие инструменты,
предназначенные непосредственно для разработчиков.

\chapter{Overall Description}

\section{Product features}
Разрабатываемая система должна предоставлять пользователям возможность
просматривать новости и статьи, делиться ими в
социальных сетях и на других площадкахи. Для
зарегистрированных пользователей должны быть доступно подключение рассылки и оформление подписок "РБК Comfort", "РБК Pro".

\section{User characteristics}


\section{Operation environment}
"РБК Недвижимость" может работать в любой системе, имеющей браузер, поддерживающий Javascript, Cookies, и рекомендуемое подключение к Интернету (более 5 Мбит/с). Например – Windows, Mac, Linux, Android, IOS.

\section{Design and Implementation Constraints}

\section{Assumptions and Dependencies}

$<$List any assumed factors (as opposed to known facts) that could affect the 
requirements stated in the SRS. These could include third-party or commercial 
components that you plan to use, issues around the development or operating 
environment, or constraints. The project could be affected if these assumptions 
are incorrect, are not shared, or change. Also identify any dependencies the 
project has on external factors, such as software components that you intend to 
reuse from another project, unless they are already documented elsewhere (for 
example, in the vision and scope document or the project plan).$>$

\section{Constraints}


\chapter{System Features}
$<$This template illustrates organizing the functional requirements for the 
product by system features, the major services provided by the product. You may 
prefer to organize this section by use case, mode of operation, user class, 
object class, functional hierarchy, or combinations of these, whatever makes the 
most logical sense for your product.$>$

\section{System Feature 1}
$<$Don’t really say “System Feature 1.” State the feature name in just a few 
words.$>$

\subsection{Description and Priority}
$<$Provide a short description of the feature and indicate whether it is of 
High, Medium, or Low priority. You could also include specific priority 
component ratings, such as benefit, penalty, cost, and risk (each rated on a 
relative scale from a low of 1 to a high of 9).$>$

\subsection{Stimulus/Response Sequences}
$<$List the sequences of user actions and system responses that stimulate the 
behavior defined for this feature. These will correspond to the dialog elements 
associated with use cases.$>$

\subsection{Functional Requirements}
$<$Itemize the detailed functional requirements associated with this feature.  
These are the software capabilities that must be present in order for the user 
to carry out the services provided by the feature, or to execute the use case.  
Include how the product should respond to anticipated error conditions or 
invalid inputs. Requirements should be concise, complete, unambiguous, 
verifiable, and necessary. Use “TBD” as a placeholder to indicate when necessary 
information is not yet available.$>$


REQ-1:	REQ-2:

\section{System Feature 2 (and so on)}


\chapter{Other Nonfunctional Requirements}

\section{Performance Requirements}
$<$If there are performance requirements for the product under various 
circumstances, state them here and explain their rationale, to help the 
developers understand the intent and make suitable design choices. Specify the 
timing relationships for real time systems. Make such requirements as specific 
as possible. You may need to state performance requirements for individual 
functional requirements or features.$>$

\section{Safety Requirements}
$<$Specify those requirements that are concerned with possible loss, damage, or 
harm that could result from the use of the product. Define any safeguards or 
actions that must be taken, as well as actions that must be prevented. Refer to 
any external policies or regulations that state safety issues that affect the 
product’s design or use. Define any safety certifications that must be 
satisfied.$>$

\section{Security Requirements}
$<$Specify any requirements regarding security or privacy issues surrounding use 
of the product or protection of the data used or created by the product. Define 
any user identity authentication requirements. Refer to any external policies or 
regulations containing security issues that affect the product. Define any 
security or privacy certifications that must be satisfied.$>$

\section{Software Quality Attributes}


\section{Business Rules}
$<$List any operating principles about the product, such as which individuals or 
roles can perform which functions under specific circumstances. These are not 
functional requirements in themselves, but they may imply certain functional 
requirements to enforce the rules.$>$


\chapter{Other Requirements}


\section{Appendix A: Glossary}
%see https://en.wikibooks.org/wiki/LaTeX/Glossary
$<$Define all the terms necessary to properly interpret the SRS, including 
acronyms and abbreviations. You may wish to build a separate glossary that spans 
multiple projects or the entire organization, and just include terms specific to 
a single project in each SRS.$>$

\section{Appendix B: Analysis Models}
$<$Optionally, include any pertinent analysis models, such as data flow 
diagrams, class diagrams, state-transition diagrams, or entity-relationship 
diagrams.$>$

\section{Appendix C: To Be Determined List}
$<$Collect a numbered list of the TBD (to be determined) references that remain 
in the SRS so they can be tracked to closure.$>$

\end{document}
